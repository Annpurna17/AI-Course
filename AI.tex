\documentclass[12pt]{article}
\title{Summary of article of philosophy of artificial intelligence}
\author{Annpurna Matawale\\annpurnamatawale@gmail.com}
\date{ july 2021}


\begin{document}
\maketitle













As Artificial intelligence overlaps many ideas with philosophy, such as action , consciousness, knowledge,epistemology(what it is reasonable to say about the world), and even free will,artificial intelligence has deeper scientific links with philosophy than other science.

artificial intelligence philosophy tries to respond to the following questions: is it possible for a machine to think like a human being ? Is it capable for solving any problem that a person could solve through reasoning ? Is there a difference between human and computer intelligence? Is it possible for a computer to have a mind , mental states, and consciousness in the same way that person does?







The main position of most AI researcher is summarised in his frist statement , which came in a proposed for the dartmouth  workshop in 1956: "Every part of learning or any other feature of intelligence may be so clearly characterised that a machine can be created to simulate it."

Arguments opposing the basic concept must demonstrate that developing a workable AI system is unfeasible due to some practical limit to computers capacities or that there is some unique property of the human mind that is required for intelligent behaviour but cannot be copied by a machine.
\\"Aerospace engineering texts do not define the objective of their field as making machines that fly so just like pigeons that they can fool to her doves" argue stuart j.russell and peter norvig.

Turing test was proposed " If a machine act as intelligently as human being , it is as intelligent as human being."
Intelligent agent : an agent is something which perceives and act in an environment.
\\Searle points out that a computer may theoritically simulate anything , thus extending the definition to its limit leads to the conclusion that any process can technically be termed computation.
\\philosopher Hubert Dreyfus invented the term "physiological assumption" to express another variation of this position :" the mind can be understood as a device functioning on bits of information according to formal rules."
Newell, simon,and dreyfus discussed "symbols" that were word like and high level symbols that directly corresponded to objects in the environment, such as <dog> and <tail>.
\paragraph{}
Turing had predicted dreyfus's argument in his 1950 study computing hardware and intelligence, which he labelled the "argument from the informality of conduct". in response, turing claimed that just because we don't know the rules that govern complex behaviour doesn't mean they don't exists.
\\Natural selection , according to daniel dennett, cannot retain a property of an animal that has no effect on the animals behaviour , hence awareness cannot be formed by natural selection.
\paragraph{}
The philosophical theories are are useful to artificial intelligence (AI) only if they do not preclude human level artificial system and provide a foundation for designing systems that have beliefs, reason, and plan according to AI . understanding common sense knowledge and abilities is a major issue for both AI and philosophy.  
\paragraph{}
Some scholars think that the AI community's dismissive attitude towards philosophy is harmful.
A group of concerns principally concerned with whether an AI is achievable - that is whether an intelligent thinking computer can be built. Its also debatable whether humans and other animals should be considered machines (say, computational robot).

















\end{document}