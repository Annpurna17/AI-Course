\documentclass[a4 paper]{article}
\usepackage[left=2cm, right=2cm,top=2cm]{geometry}
\title{Summary of Moravec's Paradox}
\author{Annpurna Matawale\\annpurnamatawale@gmail.com \\5th sem biomedical}
\date{july 2021}


\begin{document}
\maketitle
There is a discovery in field of AI, called \textbf{Moravec's Paradox}. Paradox is an absurd or contradictory statement that proves well founded or true. It observes that tasks that humans find difficult are simple for AI to learn . that is , when compared to simple sensorimotor skills that humans possess naturally.This AI paradox was described and discussed in the 1980s by Hans Moravec, Rodney Brooks, Marvin Minsky, and others. According to Moravec:\textbf{"It is comparatively easy to make computers exhibit adult level performance and difficult or impossible to give them the skills of a one year old"}.In other words for the computers the complex is easy and the easy is complex.The most difficult human skills to reverse engineer , according to Minsky , are those that are \textbf{unconscious}." We're more aware of simple process that dont work well than complex one that perform flawlessly," he wrote , adding "in general we are least aware of what our minds do well".

\paragraph{\textbf{The biological basis of human skills:}}
\paragraph{}
The explanation behind Moravecs Paradox revolves around \textbf{evolution, understanding, and perception}. 
\\For a start , the skills that we define as simple -those we learn, instinctively- are product of year and years of evolution . So, while they may appear simple , its only because of billions of year worth of tuning.Researcher look for the explanation in theory of evolution - our unconscious skills were developed and optimized during the \textbf{natural selection process}
, over millions of year of evolution.\\ In short , the skills that humans have acquired recently in their history are easier to teach computers, but our skills get harder to teach as they go further back in the evolutionary history of human and animals.
\\ Argument would be:
\\1. Reverse engineering of a human skills is proportional to the amount of time after its revolution.\\
2. The older the skills is , the more unconscious it become and hence lesser the effort required to perform it.\\
3. The more effortless a skill appear , the more difficult it is to reverse engineer.
\paragraph{}
\textbf{Moravec's Paradox and the AI of the past}
\paragraph{}
The history of AI has seen an impact from Moravec's paradox. In fact , its arguably a factor that held back development and contributed to the AI effect.\\ The AI effect is a phenomenon that has seen AI powered tools lose their AI label overtime, due to not being true intelligence. Moravec's paradox could have contributed to this . that is the reason these tools lost their intelligent status is that tha task it does are simple , once you break then down.\\ No matter how good AI tools and programs got at games and \textbf{logic} thanks to Moravec's paradox , they couldn't complete basic human task. How could anything that can't replicate the behaviour of a toddler be \textbf{intelligent}. Their optimis stemmed in part from the fact that they had been successful atvwriting programs that used logic, solved algebra and geometry problems and played games like checker and chess. 
\paragraph{}
Linguist and cognitive scientist Steven pinker considers this the main lesson uncovered by AI researchers. \\ "The mental abilities of a four year old we take for granted - recognizing a face , lifting a pencil , walking across a room, answering a question - in fact solve some of the hardest engineering problems ever conceived . The gardeners , receptionists and cooks are secure in their jobs for decades to come". 




\end{document}