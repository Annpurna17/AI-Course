\documentclass[a4 paper]{article}
\usepackage[left=2cm, right=2cm,top=2cm]{geometry}
\title{summary of Godel's incompleteness therom}
\author{Annpurna Matawale\\annpurnamatawale@gmail.com}
\date{july 2021}


\begin{document}
\maketitle

Kurt Godel's gives two theorems in 1931 . Godel's incompleteness theorems are two theorems of mathematical logic, that are concerned with the limits of provability in formal axiomatic theories. These theory is important both in mathematical logic and in the philosophy of mathematics.
\\The theorems are widely , but not universally interpreted as showing that Hilbert's program to find a \textbf{consistent set of axioms } for all mathematics is impossible.
\paragraph{}
For any such consistent formal system , there will always be statements about natural numbers that are unprovable within system.
Formal system:\textbf{completeness, consistency, and effective axiomatization}
\\
\textbf{Completeness: } A set of axioms is (syntactically, or negation) complete , if for any statement in the axioms language,that statement or its negation is provable from the axioms.
\\
\textbf{Consistency:} A set of axioms is (simply) consistent if there is no statement such that both the statement and its negation are provable from the axioms, and inconsistent otherwise.
\paragraph{\textbf{1. FIRST INCOMPLETENESS THEOREMS:}}

"Any consistent formal system F that can do a certain amount of elementry arthmetic is incomplete; thst is there are proposition of F's language that cannot be proved or disprove in F."
\\
The theorems unprovable statement GF is often referred to as \textbf{"the Godel's sentence} for system F.Each system that is effectively formed has its own Godel sentence.Godel phrase in its \textbf{syntactic} form,the Godel sentence is intented to make an indirect reference to itself.
\paragraph{\textbf{2. SECOND INCOMPLETENESS THEOREMS:}}
"Assume F is a consistent formalized system which contain elementary arithmetic . then F$\forall  $ Cons(F).For each formal system F containing basic arithmetic , it is possible to canonically dafine a formula Cons(F)expressing the consistency of F. The formula express the property that "there does not exist a natural number coding a formal derivation within the system F whose conclusion is a syntactic contradiction." The syntactic contradiction is often taken to be "0-1" in which case Cons(F) state there is no natural number that codes a derivation of 0-1 from the axioms of F. 
\\
The second incompleteness theorem extension of the first , show that the system can not demonstrate own consistency employing a diagonal argument godel incompleteness theorem were the first of several closely related theorems on the limitation of \textbf{formal system} .
\\
The proof of second incompleteness theorem is obtained by formalizing the first incompleteness theorem within the system F itself.
\paragraph{}

compared to the theorem stated in Godels 1931 paper , many contemporary statements of the incompleteness theorems are more general in two ways.the incompleteness theorem results affects the philosophy of mathematics , particularly  version of formalism , which use a single systemof formal logic to define their principles.

There are various philosophers like Hilary putnam suggested that while Gdels theorem can not be applied to humans since they make mistake and are therefor consistent , it may be applied to the humans faculty of science or mathematical in general. Douglas Hofstadter refers godels  theorems as an example of what he calls a strange loop, a hierarchical , self referential structure existing within an aximatic formal system.
\\
The incompleteness theorems apply to formal system that are of sufficient complexity to express the basic arithmetic of natural numbers and which are consistent and effectively axiomatized . 





\end{document}
